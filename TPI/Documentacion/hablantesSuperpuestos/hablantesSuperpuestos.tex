\documentclass{article}
\usepackage{listings}
\usepackage[utf8]{inputenc}
\input{../latex/Algo1Macros}

\usepackage{listings}
\usepackage{xcolor}

\definecolor{codegreen}{rgb}{0,0.6,0}
\definecolor{codegray}{rgb}{0.5,0.5,0.5}
\definecolor{codepurple}{rgb}{0.58,0,0.82}
\definecolor{backcolour}{rgb}{0.95,0.95,0.92}

\lstset
{
    language=Caml,
    backgroundcolor=\color{backcolour},
    commentstyle=\color{codegreen},
    keywordstyle=\color{magenta},
    numberstyle=\tiny\color{codegray},
    stringstyle=\color{codepurple},
    basicstyle=\ttfamily\footnotesize,
    breakatwhitespace=false,
    breaklines=true,
    captionpos=b,
    keepspaces=true,
    numbers=left,
    numbersep=5pt,
    showspaces=false,
    showstringspaces=false,
    showtabs=false,
    tabsize=2
}

\begin{document}

    \section*{Hablantes Superpuetos}

    \subsection*{valorAbsoluto}

    \begin{minipage}{0.70\textwidth}
        \lstinputlisting{./code/valorAbsoluto.cpp}
    \end{minipage}
    \hfill
    \begin{minipage}{0.25\textwidth}
        \resizebox{0.21\textwidth}{!}{%
        \begin{tabular}{|c|c}
            $               $ &$   $\\
            $c_1            $ &$ 1 $\\
            $c_2            $ &$ 1 $\\
            $               $ &$   $\\
            $               $ &$   $\\
            $               $ &$   $\\
        \end{tabular}}
    \end{minipage}

    \begin{itemize}
        \item $T_{valorAbsoluto}(n) = c_1 + c_2$
        \item $T_{valorAbsoluto}(n) \in O(1)$
    \end{itemize}

    \subsection*{hablantesSuperpuestos}

    \begin{minipage}{0.70\textwidth}
        \lstinputlisting{./code/hablantesSuperpuestos.cpp}
    \end{minipage}
    \hfill  
    \begin{minipage}{0.25\textwidth}
        \resizebox{0.476\textwidth}{!}{%
        \begin{tabular}{|c|c}
            $                           $ & $               $\\
            $c^{\prime}_1               $ & $1              $\\
            $c^{\prime}_2               $ & $n              $\\
            $c^{\prime}_3               $ & $1              $\\
            $c^{\prime}_4               $ & $m + 1          $\\
            $c^{\prime}_5               $ & $m              $\\
            $c^{\prime}_6               $ & $m              $\\
            $c^{\prime}_7               $ & $m \cdot (n + 1)$\\
            $c^{\prime}_8               $ & $m \cdot (n - 1)$\\
            $c^{\prime}_9               $ & $m \cdot (n - 1)$\\
            $                           $ & $               $\\
            $c^{\prime}_{10}            $ & $m              $\\
            $c^{\prime}_{11}            $ & $m              $\\
            $c^{\prime}_{12}            $ & $m              $\\
            $c^{\prime}_{13}            $ & $m              $\\
            $                           $ & $               $\\
            $c^{\prime}_{14}            $ & $m              $\\
            $                           $ & $               $\\
            $                           $ & $               $\\
            $c^{\prime}_{15}            $ & $m              $\\
            $c^{\prime}_{16}            $ & $m              $\\
            $                           $ & $               $\\
            $c^{\prime}_{17}            $ & $m \cdot n      $\\
            $                           $ & $               $\\
            $c^{\prime}_{18}            $ & $m              $\\
            $                           $ & $               $\\
            $                           $ & $               $\\
            $                           $ & $               $\\
        \end{tabular}}
    \end{minipage}

    \begin{itemize}
        \item $n = \longitud{r[0].first}$
        \item $m = \longitud{r}$
        \item $T_{hablantesSuperpuestos}(m, n) = c^{\prime}_1 +
                                              c^{\prime}_2 \cdot n +
                                              c^{\prime}_3 +
                                              c^{\prime}_4 \cdot (m + 1) +
                                              c^{\prime}_5 \cdot m +
                                              c^{\prime}_6 \cdot m +
                                              c^{\prime}_7 \cdot m \cdot (n + 1) +
                                              c^{\prime}_8 \cdot m \cdot (n - 1) + \\
                                              c^{\prime}_9 \cdot m \cdot (n - 1) +
                                              c^{\prime}_{10} \cdot m +
                                              c^{\prime}_{12} \cdot m +
                                              c^{\prime}_{13} \cdot m +
                                              c^{\prime}_{14} \cdot m +
                                              c^{\prime}_{15} \cdot m +
                                              c^{\prime}_{16} \cdot m +
                                              c^{\prime}_{17} \cdot m \cdot n +
                                              c^{\prime}_{18} \cdot m $
        \item $T_{hablantesSuperpuestos}(m, n) \in O(m \cdot n)$
    \end{itemize}

\end{document}
